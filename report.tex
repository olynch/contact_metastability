\documentclass{scrartcl}
\usepackage{latexrc/macros}
\title{Report on Metastability for the Contact Process}
\author{Owen Lynch \and Kacper Urbansky}
\begin{document}

\maketitle

\section{Overview}

The essential idea of metastability is that, as some parameter goes to infinity, we can coarse grain a Markov process into something that looks like the Markov process in \fref{fig:metastability_nutshell}.

\begin{figure}[h!]
  \centering

  \caption{Metastability in a Nutshell}
  \ref{fig:metastability_nutshell}
\end{figure}

The system begins in the so-called ``metastable state'', and after an exponential random time, transitions into the stable state. The stable state is either absorbing, or close-to absorbing; in the case that we will talk about in this paper, the stable state is absorbing.

There are two key features of

\section{Review of Contact Process}

\section{Hitting Time of Stable State}

\section{Almost-Ergodicity up to Hitting Time}

\end{document}
